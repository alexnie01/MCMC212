\documentclass{article}
\usepackage{amssymb, amsmath}
\usepackage[margin=1in]{geometry}
\newcommand{\tot}[2]{\frac{d\,#1}{d\,#2}}
\title{Physics 212 Problem Set 1}
\author{Alex Nie}
\date{September 15, 2016}
\begin{document}
\maketitle

\section*{Problem 1}
\subsection*{$H_0$}
\subsubsection*{a)}
Using natural units, we can convert between the $\frac{1}{time}$ units of $H_0$ to $Mpc^{-1}$. 

\[H_0 = \frac{70 km}{s \cdot Mpc}\cdot\frac{1000m}{km}\cdot\frac{1\, c}{3\times 10^8\frac{m}{s}}=2.33\times 10^{-4}Mpc^{-1}\]

\subsubsection*{b)}
\[H_0 = \frac{70\,km}{s\cdot Mpc}\cdot\frac{1\,Mpc}{3.08\times 10^{19}\,km}\cdot\frac{3.15\times 10^{16}\,s}{1\,Gyr} = 7.13\times 10^{-2}\,Gyr^{-1}\]

\subsection*{$\rho_{crit}$}
\subsubsection*{a)}
We use $G=6.674\times10^{-8}\frac{cm^3}{g\cdot s^2}$. Since this already has units of $\frac{g}{cm^3}$ we simply need to re-express $H_0$ in $\frac{1}{s}$.

\[H_0 = \frac{70\,km}{s\cdot Mpc}\cdot\frac{1\,Mpc}{3.08\times 10^{19}\,km}=2.27\times 10^{-18}s^{-1}\]

Therefore
\[\rho_{crit} = \frac{3 (2.27\times 10^{-18}s^{-1})^2}{8\pi\cdot 6.674\times 10^{-8}\frac{cm^3}{g\cdot s^2}}=9.21\times 10^{-30}\frac{g}{cm^3}\]
\subsubsection*{b)}
Since $1\,eV = q_e \times \frac{J}{C} = 1.6\times 10^{-19} J$, we seek to convert the joules into units of $\frac{g}{c^2}$.

\[1\,\frac{eV}{c^2} = \frac{1.6\times 10^{-19} J}{c^2}\left(\frac{c}{3\times 10^8 \frac{m}{s}}\right)^2\left(\frac{kg\cdot \frac{m^2}{s^2}}{J}\right)\left(\frac{1000\,g}{1\,kg}\right) 1.78\times 10^{-33}g\]

Therefore, using our answer from part a),
\[\rho_{crit} = \frac{9.21\times 10^{-30}\,g}{cm^3}\cdot\frac{1\frac{eV}{c^2}}{1.78\times 10^{-33}\,g}=5.18\times 10^3 \frac{eV}{cm^3}\]

\subsubsection*{c)}
We use the fact that in natural units, $\hbar = 6.582\times 10^{-16}\,eV\cdot s$. Then
\[\rho_{crit} = \frac{5.18\times 10^3 eV}{cm^3}\left(\frac{10^2\,cm}{1m}\right)^3\left(\frac{3\times 10^8 \frac{m}{s}}{1\,c}\right)^3\left(\frac{6.582\times 10^{-16}\,eV\cdot s}{\hbar}\right)^3\left(\frac{1 GeV}{10^9 eV}\right)^4 = 3.99\times 10^{-47}GeV^4\]

\section*{Problem 2}
\subsection*{a)}
From class, we showed that
\[\left(\frac{H}{H_0}\right)^2 =\Omega_m (1+z)^3+\Omega_{\Lambda}\]

Substituting in $H = \frac{\dot{a}}{a}$ we have
\begin{align*}
\frac{1}{a}\tot{a}{t} &= H_0\sqrt{\Omega_m(1+z)^3+\Omega_{\Lambda}}\\
\frac{da}{a\,H_0\sqrt{\Omega_m a^{-3}+\Omega_{\Lambda}}} &=\,dt
\end{align*}

We wish to integrate from $t=0$ to now, $t_age$ in terms of time and from $a=0$ to $a=1$ in terms of $a$. Then
\[t_age = \frac{1}{2.27\times 10^{-18} s^{-1}}\int_0^1 \frac{da}{a\sqrt{.3 a^{-3} + .7}}=4.25\times 10^{17}s=13.5Gyr\]
\subsection*{b)}
As stated in class, the current particle horizon is given by 
\[\eta = \int_0^t \frac{c\,dt'}{a(t')}\]

As we do not $a$ as an explicit function of $t$ we change variables. In particular,
\begin{align*}
\left(\frac{H}{H_0}\right)^2 &= \left(\frac{1}{a}\tot{a}{t}\right)^2 = \Omega_m\,a^{-3} + \Omega_\Lambda\\
dt &= \frac{da}{H_0\,a\,\sqrt{\Omega_ma^{-3}+\Omega_{\Lambda}}}
\end{align*}

Substituting into the definition of $\eta$, 
\begin{align*}
\eta &= \int_0^1 \frac{c\,da}{H_0\,a^2\,\sqrt{\Omega_ma^{-3}+\Omega_\Lambda}} \\
&= \frac{3\times 10^8 \frac{m}{s}}{2.27\times 10^{-18}s^{-1}}\left(\frac{1\,ly}{9.46\times 10^{15} m}\right)\int_0^1 \frac{da}{a^2\sqrt{(.3)a^{-3}+.7}}\\
&=4.62\times 10^{10}\,ly
\end{align*}

In the limit that $a\rightarrow\infty$,
\[\eta = \int_0^{\infty}\frac{c\,da}{a^2 H_0 \sqrt{\Omega_m a^{-3} + \Omega_{\Lambda}}}= 6.21\times 10^{10}\,ly\]

so the particle horizon (in the present co-moving coordinate system) approaches this limit as the universe evolves.

\section*{Problem 3}
From class we have that $G_{\mu\nu} = R_{\mu\nu}-\frac{1}{2}g_{\mu\nu}R$.
For a general FRW metric
\[ds^2 = -c^2dt^2 + a^2(t)\left(\frac{dr^2}{1-kr^2}+r^2(d\theta^2+\sin^2\theta\,d\phi^2)\right)\]

Therefore we can read off the metric
\[g = \begin{bmatrix} -c^2&0&0&0\\0 &\frac{a^2(t)}{1-kr^2}&0&0\\0&0&a^2(t)r^2&0\\0&0&0&a^2(t)r^2\sin^2\theta\end{bmatrix}\]

We now calculate Christoffel coefficients using the Lagrangian method. For a given metric
\[L = g_{\mu\nu}\dot{x}^\mu\dot{x}^\nu = -c^2\dot{t}^2 + a^2(t)\left(\frac{\dot{r}^2}{1-kr^2}+r^2(\dot{\theta}^2+\sin^2\theta\dot{\phi}^2)\right)\]

Where dots denote derivatives with respect to conformal time.

Starting with $r$, we have
\[\frac{\partial L}{\partial r} = \frac{2kra^2(t)\dot{r}^2}{(1-kr^2)^2}+2a^2(t)r(\dot{\theta}^2+\sin^2\theta\dot{\phi}^2)\]
\[\frac{\partial L}{\partial \dot{r}} = \frac{2a^2(t)\dot{r}}{1-kr^2}\]
\[\frac{d}{ds}\left(\frac{\partial L}{\partial \dot{r}}\right)= \frac{2a^2(t)\ddot{r}}{1-kr^2}+\frac{4a(t)\dot{a}\dot{t}\dot{r}}{1-kr^2} + \frac{2a^2(t)kr\dot{r}^2}{(1-kr^2)^2} \]

Using the chain rule, we have $\frac{da}{dt} = \frac{\dot{a}}{a}$. Setting our equations equal, we obtain,
\[\ddot{r} = r(1-kr^2)(\dot{\theta}^2+\sin^2\theta\dot{\phi}^2)-\frac{2\dot{a}\dot{t}\dot{r}}{a}\]

Therefore,
\[\Gamma^1_{01}=\Gamma^1_{10} = \frac{\dot{a}}{a^2}\]
\[\Gamma^1_{22} = -r(1-kr^2)\]
\[\Gamma^1_{33} = -r\sin^2\theta(1-kr^2)\]

Similarly for $\theta$, 
\[\frac{\partial L}{\partial \theta} = 2a^2(t)r^2\sin\theta\cos\theta\dot{\phi}^2\]
\[\frac{d}{ds}\left(\frac{\partial L}{\partial \dot{\theta}}\right) = \frac{d}{ds}\left(2a^2(t)r^2\dot{\theta}\right) = 2a^2(t)r^2\ddot{\theta} + 4r^2a(t)\dot{a}\dot{t}\dot{\theta}+4a^2(t)r\dot{r}\dot{\theta}\]
\[\ddot{\theta} = \sin\theta\cos\theta\dot{\phi}^2 -\frac{2\dot{a}\dot{t}\dot{\theta}}{a^2}-\frac{2\dot{r}\dot{\theta}}{r}\]

Then
\[\Gamma^2_{02}=\Gamma^2_{20} = \frac{\dot{a}}{a}\]
\[\Gamma^2_{12}=\Gamma^2_{21} = \frac{1}{r}\]
\[\Gamma^2_{33} = -\sin\theta\cos\theta\]

For $\phi$,
\[\frac{\partial L}{\partial \phi} = 0\]
\[\frac{d}{ds}\left(\frac{\partial L}{\partial\dot{\phi}}\right) = \frac{d}{ds}\left(2a^2(t)r^2\sin^2\theta\dot{\phi}\right)=4r^2\sin^2\theta a(t)\dot{a}\dot{t}\dot{\phi} + 4a^2(t)r\sin^2\theta\dot{r}\dot{\phi} + 4a^2(t)r^2\sin\theta\cos\theta\dot{\theta}\dot{\phi} + 2a^2(t)r^2\sin^2\theta\ddot{\phi}\]
\[\ddot{\phi} = -\frac{2\dot{a}\dot{t}\dot{\phi}}{a}-\frac{2\dot{r}\dot{\phi}}{r}-2\cot\theta\dot{\theta}\dot{\phi}\]

Therefore
\[\Gamma^3_{03}=-\Gamma^3_{30} = \frac{\dot{a}}{a}\]
\[\Gamma^3_{13}=-\Gamma^3_{31} = \frac{1}{r}\]
\[\Gamma^3_{23}=-\Gamma^3_{32} = \cot\theta\]

Finally for $t$,
\[\frac{\partial L}{\partial t} = 2a(t)\dot{a}\left(\frac{\dot{r}^2}{1-kr^2}+r^2(\dot{\theta}^2+\sin^2\theta\dot{\phi}^2)\right)\]
\[\frac{d}{ds}\left(\frac{\partial L}{\partial \dot{t}}\right) = \frac{d}{ds}\left(-2c^2\dot{t}\right) = -2c^2\ddot{t}\]
\[\ddot{t} =-\dot{a}\left(\frac{\dot{r}^2}{1-kr^2}+r^2(\dot{\theta}^2+\sin^2\theta\dot{\phi}^2)\right)\]

where we have set $c=1$. Then
\[\Gamma^0_{11} = \frac{a(t)\dot{a}}{1-kr^2}\]
\[\Gamma^0_{22} = a(t)\dot{a}r^2\]
\[\Gamma^0_{33} = a(t)\dot{a}r^2\sin^2\theta\]

We now calculate the components of the Ricci tensor and scalar.

\[R_{00} = \partial_\mu\Gamma^\mu_{00}-\partial_0\Gamma^\mu_{0\mu}+\Gamma^\mu_{\nu\mu}\Gamma^\nu_{00}-\Gamma^{\mu}_{\nu0}\Gamma^{\nu}_{0\mu}\]
We immediately see that the first and third terms are $0$ since no Christoffel coefficient with lower indices $00$ has a non-zero component.

Then
\begin{align*}
R_{00} &= -\partial_t\left(\frac{\dot{3a}}{a}\right)-2(\Gamma^{1}_{10}\Gamma^1_{01}-\Gamma^2_{02}\Gamma^2_{20}-\Gamma^3_{03}\Gamma^3_{30})\\
&=-\frac{3\ddot{a}}{a}-\frac{6\dot{a}^2}{a^2}+\frac{6\dot{a}^2}{a^2}\\
&= \frac{-3\ddot{a}}{a}
\end{align*}
\begin{align*}
R_{11} &= \partial_\mu\Gamma^\mu_{11}-\partial_1\Gamma^\mu_{1\mu}+\Gamma^\mu_{\nu\mu}\Gamma^\nu_{11}-\Gamma^{\mu}_{\nu1}\Gamma^{\nu}_{1\mu}\\
&= \partial_0\Gamma^0_{11}-\partial_1\Gamma^0_{10}-\Gamma^1_{01}\Gamma^1_{10}-\Gamma^2_{12}\Gamma^2_{21}-\Gamma^3_{13}\Gamma^3_{31}\\
&=\partial_t(\frac{a\dot{a}}{1-kr^2})-\frac{1}{r^2}-\frac{1}{r^2}\\
&= \frac{2\dot{a}^2+\ddot{a}+2k}{1-kr^2}
\end{align*}

\begin{align*}
R_{22} &= \partial_\mu\Gamma^\mu_{22}-\partial_2\Gamma^\mu_{2\mu}+\Gamma^\mu_{\nu\mu}\Gamma^\nu_{22}-\Gamma^{\mu}_{\nu2}\Gamma^{\nu}_{2\mu}\\
&=\partial_0\Gamma^0_{22}+\partial_1\Gamma^1_{22}-\Gamma^\mu_{0\mu}\Gamma^0_{22}-\Gamma^\mu_{1\mu}\Gamma^1_{22}-\Gamma^0_{22}\Gamma^2_{20}\\
&= \partial_t(a\dot{a}r^2)+\partial_r(-r(1-kr^2))-\frac{3\dot{a}}{a}(a\dot{a}r^2)-\frac{2}{r}(-r(1-kr^2))-(a\dot{a}r^2)(\frac{\dot{a}}{a})\\
&=r^2(a\ddot{a}+2\dot{a}^2+2k)
\end{align*}

\begin{align*}
R_{33} &= \partial_\mu\Gamma^\mu_{33}-\partial_3\Gamma^\mu_{3\mu}+\Gamma^\mu_{\nu\mu}\Gamma^\nu_{33}-\Gamma^{\mu}_{\nu3}\Gamma^{\nu}_{3\mu}\\
&=\partial_0\Gamma^0_{33}+\partial_1\Gamma^1_{33}+\partial_2\Gamma^2_{33}+\Gamma^\mu_{\nu\mu}\Gamma^\nu_{33}-\Gamma^\mu_{\nu3}\Gamma^\nu_{3\mu}\\
&= \partial_t(a\dot{a}r^2\sin^2\theta)+\partial_r(-r\sin^2\theta(1-kr^2))+\partial_\theta(-\sin\theta\cos\theta)\\
&\,+\Gamma^\mu_{1\mu}\Gamma^1_{33}+\Gamma^\mu_{2\mu}\Gamma^2_{33}+\Gamma^\mu_{0\mu}\Gamma^0_{33}-\Gamma^3_{03}\Gamma^0_{33}-\Gamma^3_{13}\Gamma^1_{33}-\Gamma^3_{23}\Gamma^2_{33}\\
&=r^2(a\ddot{a}+2\dot{a}^2+2k)\sin^2\theta
\end{align*}

We then see that
\[R = g^{\mu\nu}R_{\mu\nu} = -R_{00}+\frac{1-kr^2}{a^2}R_{11}+\frac{1}{a^2r^2}R_{22}+\frac{1}{a^2r^2\sin^2\theta}R_{33}=6\left(\frac{\ddot{a}}{a}+\left(\frac{\dot{a}}{a}\right)^2+\frac{k}{a^2}\right)\]

Substituting into the Einstein equation,
\[G^0_0 = R^0_0-\frac{1}{2}g^0_0 R = -\frac{3\ddot{a}}{a}-\frac{1}{2}\left(6\left(\frac{\ddot{a}}{a}+\left(\frac{\dot{a}}{a}\right)^2+\frac{k}{a^2}\right)\right)=-\frac{3}{a^2}\left(\left(\frac{\dot{a}}{a}\right)^2+k\right)\]

For $i\not=j$, $g^i_j=R^i_j=0$ due to the symmetry of the problem. Therefore,
\[G^i_j = \left(R^i_j-\frac{1}{2}g^i_jR\right)\delta^j_i = -\frac{1}{a^2}\left(2\frac{\ddot{a}}{a}-\left(\frac{\dot{a}}{a}\right)^2+k\right)\delta^i_j\]
\section*{Problem 4}
\subsection*{a)}
The comoving distance is given by
\[r = \int_t^{t_0} \frac{c dt'}{a(t')}\]
Since we would like to know $r$ as a function of $z$, we substitute variables.
\begin{align*}
a(t) = \left(\frac{t}{t_0}\right)^{\frac{2}{3}}&=\frac{1}{1+z}\\
\frac{t}{t_0} &= (1+z)^{-\frac{3}{2}}\\
dt &= -\frac{3}{2}t_0(1+z)^{-\frac{5}{2}}dz
\end{align*}
Substituting into our integral, we have at $t=t_0$, $z=0$. Then
\begin{align*}
r &= \int_0^z \frac{3ct_0\,dz}{2(1+z)^{\frac{3}{2}}}\\
&=3ct_0\left(1-\frac{1}{\sqrt{1+z}}\right)
\end{align*}
\subsection*{b)}
We can read off the differential co-moving volume from the spatial components of the metric. Letting $a(t)=1$ in the co-moving coordinate system,
\[dV = \prod_{i=1}^3 \sqrt{|g_{ii}|}dx^i = S(r)^2\sin\theta\,dr\,d\theta\,d\phi\]
For a flat universe, $S(r)=r$. Next, a unit steradian is defined as the unit solid angle. If we imagine a differential rectangular patch which is subtended by angles of $d\theta$ and $d\phi$, then its side lengths are given by $d\theta$ and $\sin\theta\,d\phi$ so $d\Omega = \sin\theta\,d\theta\,d\phi$. Therefore, the co-moving volume per unit steradian per unit redshift is
\[\frac{dV}{d\Omega\,dz} = r^2\tot{r}{z}\]
Substituting in our expression from part a),
\[\frac{dV}{d\Omega\,dz} = \frac{27c^3t_0^3}{2(1+z)^\frac{3}{2}}\left(1-\frac{1}{\sqrt{1+z}}\right)^2\]
\section*{Problem 5}
\subsection*{a)}
Equating the expressions for densities at an arbitrary time, we obtain
\begin{align*}
\rho_{m,0} &= \rho_{r,0}\\
(1+z)^3\rho_{m,0} &= (1+z)^4\rho_{r,0}\\
\Omega_m\rho_{crit} &= (1+z)\Omega_r \rho_{crit}\\
z&=\frac{\Omega_m}{\Omega_r}-1\\
&= \frac{.14}{4\times 10^{-5}}-1\\
&= 3.499\times 10^{3}
\end{align*}

Since $a=\frac{1}{1+z}$ then
\[z=\frac{\Omega_m}{\Omega_r}-1\implies a_{eq}=\frac{\Omega_r}{\Omega_m}\]
\subsection*{b)}
Using the definition given in the problem statement,
\begin{align*}
H^2 &= \frac{8\pi G \rho}{3}\\
&= \frac{8\pi G}{3}(\rho_m+\rho_r)\\
&= \frac{8\pi G}{3}(1+z)^3(\rho_{m,0}+(1+z)\rho_{r,0})\\
&= \frac{8\pi G}{3}(1+z)^3\rho_{crit}(\Omega_m+(1+z)\Omega_r)\\
&= (1+z)^3 H_0^2(\Omega_m+(1+z)\Omega_r)\\
&= (1001)^3\left(100\frac{h\cdot km}{s\cdot Mpc}\right)^2\left(\frac{.14}{h^2}+\frac{(1001)(4\times 10^{-5})}{h^2}\right)\\
&= 1.80\times 10^{12}(\frac{km}{s\cdot Mpc})^2(\frac{1 Mpc}{3.08\times 10^{19}\,km})^2\\
&=1.90\times 10^{-27}s^{-2}
\end{align*}
Then $H(z=1000) = \sqrt{1.90\times 10^{-27}s^{-2}} = 4.36\times 10^{-14} s^{-1}$

The Hubble time is therefore
\[t_H(z=1000) = \frac{1}{H} = 2.29\times 10^13 s = 7.27\times 10^{5}yr\]
\subsection*{c)}
Starting at
\[\left(\frac{H}{H_0}\right)^2 = \Omega_m(1+z)^3+\Omega_r(1+z)^4\]
We can substitute in $a=\frac{1}{1+z}$ and $H = \frac{\dot{a}}{a}$ to obtain
\[dt = \frac{da}{aH_0\sqrt{\Omega_m\,a^{-3}+\Omega_r a^{-4}}}\]
Since we would like the age of the universe, we want to integrate from $t=0$ to $t_age$ and from $a=0$ to $a(z=1000) = \frac{1}{1001}$. Then
\[t_{age} = \int_0^{\frac{1}{1001}}\frac{da}{100a\sqrt{.14a^{-3}+4\times 10^{-5}a^{-4}}} = 4.45\times 10^{-7} \frac{s\cdot Mpc}{km}\]
Converting to years,
\[t_{age} = \frac{4.45\times 10^{-7}\,s\cdot Mpc}{km}\left(\frac{3.08\times 10^19\,km}{1\,Mpc}\right) = 1.37\times 10^13 s = 4.35\times 10^5 yr\]
\subsection*{d)}
For a small period of physical time, a photon travels $\Delta x = c\,dt$ of physical distance, and $\frac{c\,dt}{a(t)}$ of co-moving distance. Therefore, the total co-moving distance is
\[r=\int_0^t \frac{c\,dt}{a(t)}\]
Using the chain rule,
\[dt\left(\frac{dt}{da}\frac{da}{dt}\right) = \frac{da}{aH}\]
Then
\begin{align*}
r&=\frac{c}{H_0}\int_0^{\frac{1}{1001}}\frac{da}{a^2\sqrt{\Omega_m a^{-3}+\Omega_r a^{-4}}}\\
&=\frac{3\times 10^8 \frac{m}{s}}{100 \frac{h\cdot km}{s\cdot Mpc}}\left(\frac{3.08\times 10^{19}\,km}{1\,Mpc}\right)\int_0^{\frac{1}{1001}} \frac{da}{\sqrt{\frac{.14}{h^2}a+\frac{4\times 10^{-5}}{h^2}}}\\
&= 9.35\times 10^{24}m \left(\frac{1\,Mpc}{3.08\times 10^{22}\,m}\right)\\
&= 303 Mpc
\end{align*}
\subsection*{e)}
Repeating the integral with $a=1$ at today,
\begin{align*}
r&=\frac{3\times 10^8 \frac{m}{s}}{100 \frac{h\cdot km}{s\cdot Mpc}}\left(\frac{3.08\times 10^{19}\,km}{1\,Mpc}\right)\int_\frac{1}{1001}^1 \frac{da}{\sqrt{\frac{.14}{h^2}a+\frac{4\times 10^{-5}}{h^2}}}\\
&=4.76\times 10^{26} m\left(\frac{1\,Mpc}{3.08\times 10^{22}m}\right)\\
&=1.54\times 10^4 Mpc
\end{align*}
The co-moving distance in $d)$ is much smaller than the co-moving distance between $z=1000$ and today.
\end{document}
