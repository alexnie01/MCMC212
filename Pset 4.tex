\documentclass{article}
\usepackage{amssymb, amsmath}
\usepackage[margin=1in]{geometry}
\newcommand{\tot}[2]{\frac{d\,#1}{d\,#2}}
\title{Physics 212 Problem Set 4}
\author{Alex Nie}
\date{September 21, 2016}
\begin{document}
\maketitle

\section*{Problem 1}
\subsection*{a)}
From class we have that
\[\frac{\Delta_T^2}{\Delta_R^2} =\frac{\frac{2H_*^2}{\pi^2 M_{Pl}^2}}{\frac{H_*^4}{(2\pi)^2\dot{\Phi}^2}} = \frac{8\dot{\Phi}^2}{M_{Pl}^2H_*^2}\]
In the slow-roll approximation, we have
\begin{align*}
3H\dot{\Phi} &= V'\\
H^2 &= -\frac{8\pi G}{3}V\\
\epsilon = -\frac{\dot{H}}{H^2}
\end{align*}
Differentiating the second equation,
\begin{align*}
2H\dot{H} &= -\frac{8\pi G}{3}V'\dot{\Phi}\\
&= -\left(\frac{8\pi G}{3}\right)(3H\dot{\Phi})\dot{\Phi}\\
\dot{H} &= -4\pi G \dot{\Phi}^2
\end{align*}

Then substituting into $\epsilon$, 
\[\epsilon = -\frac{\dot{H}}{H^2} = \frac{4\pi G\dot{\Phi}^2}{H^2}\]

Then
\[\frac{\Delta_T^2}{\Delta_R^2} = \frac{2\epsilon}{\pi}\]
\subsection*{b)}
By definition, $N=\int H\,dt$ so $dN = H\,dt=\frac{H}{\dot{H}}\,dH$. Then
\[\frac{d\ln H}{dN} = \frac{1}{H}\tot{H}{N} = \frac{\dot{H}}{H^2} = -\epsilon\]
We use the chain rule through $N$ to rewrite $d\ln k$. We then have
\[\ln k = -\ln (aH) = -\ln(a_* e^N H) = -\ln a_* - N - \ln H\]
\[\tot{\ln k}{N} = -(1+\tot{\ln H}{N}) = \epsilon-1\]
Then the implicit function theorem,
\[\tot{N}{\ln k} = -\frac{1}{\epsilon-1}\]

As shown in part 1, 
\[\dot{H} = -4\pi G\dot{\phi}^2\]

Then
\[\Delta_R^2 = \frac{H_*^4}{(2\pi)^2}\left(\frac{-4\pi G}{\dot{H}}\right) = C_2 \frac{H_*^2}{\epsilon}\]
for some constant $C_2$. Therefore,

\begin{align*}
n_s-1 &= \tot{\ln \Delta_R^2}{\ln k}\\
&= 2\tot{ln H_*}{\ln k}-\tot{\ln \epsilon}{\ln k}\\
&= 2\epsilon-\frac{1}{\epsilon}\tot{\epsilon}{N}\tot{N}{\ln k}\\
&=2\epsilon-\eta
\end{align*}

Where we have used
\begin{align*}
\frac{1}{\epsilon}\tot{\epsilon}{N} &= \frac{1}{\epsilon}\tot{\epsilon}{t}\tot{t}{H}\tot{H}{N}\\
&= \frac{\dot{\epsilon}}{\epsilon H}\\
&=\eta
\end{align*}
\subsection*{c)}
From class,
\[\ln \Delta_t^2 = \ln\left(\frac{2H_*^2}{\pi M_{Pl}^2}\right) = 2\ln H_* + C_1\]

for some constant $C_1$. Then similar to part b), we have
\[n_t=\tot{\ln \Delta_t^2}{\ln k} = \tot{\ln \Delta_t^2}{N}\tot{N}{\ln k} = \frac{2\epsilon}{\epsilon-1}\approx -2\epsilon\]

\subsection*{d)}

We have that $r = \frac{2\epsilon}{\pi}$ and $n_t = 2\epsilon$ so $r=-\frac{n_t}{\pi}$.
\end{document}