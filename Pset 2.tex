\documentclass{article}
\usepackage{amssymb, amsmath}
\usepackage[margin=1in]{geometry}
\newcommand{\tot}[2]{\frac{d\,#1}{d\,#2}}
\title{Physics 212 Problem Set 2}
\author{Alex Nie}
\date{September 21, 2016}
\begin{document}
\maketitle

\section*{Problem 1}
\subsection*{a)}
For this problem we use the chain rule:
\[\tot{n_0}{v_0} = \tot{n_0}{n_d}\tot{n_d}{p_d}\tot{p_d}{p_0}\tot{p_0}{v_0}\]

In class we showed that for a given particle,
\[n = \int_{0}^{\infty} \frac{g}{h^3}\frac{4\pi p^2\,dp}{1+\exp(\frac{pc}{kT})} \]
Then taking the derivative and picking a particular time,
\[\tot{n_d}{p_d} = \frac{g}{h^3}\frac{4\pi p_d^2}{1+\exp(\frac{p_d}{kT_d})}\]

From the problem statement, we are given $p_d = p_0(1+z)$ and $T_d = \left(\frac{4}{11}\right)^{\frac{1}{3}}(1+z_d)T_0$. Moreover, as $n$ is a density, if we assume that the same total number of neutrinos remains constant after decoupling, then $n_d = n_0(1+z)^3$. Then
\begin{align*}
\tot{n_d}{p_d} &= \frac{g}{h^3}\frac{4\pi p_0^2 (1+z_d)^2}{1+\exp\left(\frac{p_0(1+z_d)c}{k\left(\frac{4}{11}\right)^{\frac{1}{3}}(1+z_d)T_0}\right)}\\
\tot{n_0}{n_d} &= \frac{1}{(1+z_d)^3}\\
\tot{p_d}{p_0} &= 1+z_d\\
\tot{p_0}{v_0} &= m
\end{align*}

Then multiplying our expressions together,
\[\tot{n_0}{v_0} = \frac{g}{h^3}\frac{4\pi m p_0^2}{1+\exp\left(\frac{p_0 c}{\left(\frac{4}{11}\right)^{\frac{1}{3}}k T_0}\right)}\]

Since the neutrinos are non-relativistic today, $p_0 = m v_0$. Substituting in,
\[\tot{n_0}{v_0} = \frac{g}{h^3}\frac{4\pi m_\nu^3v_0^2}{1+\exp\left(\frac{m_\nu v_0 c}{\left(\frac{4}{11}\right)^{\frac{1}{3}}kT_0}\right)}\]

\subsection*{b)}
For this problem we calculate the "velocity density" and divide by the number density. In particular,
\[\langle v\rangle = \frac{\int v\,dn}{\int\,dn}\]
(We can see that this produces the average velocity the terms in the numerator and denominator are densities and the term in the numerator is the first moment in a velocity distribution)

Then
\begin{align*}
\int v\,dn &= \int v\left(\tot{n}{v}\right)\,dv\\
&= \int_0^\infty \frac{g}{h^3}\frac{4\pi m_\nu^3 v^3\,dv}{1+\exp\left(\frac{m_\nu v c}{\left(\frac{4}{11}\right)^{\frac{1}{3}}k T_0}\right)}\\
&= \frac{4\pi m_v^3 g}{h^3}\left(\frac{\left(\frac{4}{11}\right)^{\frac{1}{3}}k T_0}{m_\nu c}\right)^4 \int_0^\infty \frac{u^3\,du}{1+\exp(u)}\\
&= \frac{4\pi m_\nu^3 g}{h^3}\left(\frac{\left(\frac{4}{11}\right)^{\frac{1}{3}} kT_0}{m_\nu c}\right)^4 3!\zeta(4)\left(\frac{7}{8}\right)
\end{align*}

\begin{align*}
\int\,dn &= \int \left(\tot{n}{v}\right)\,dv\\
&= \int_0^\infty \frac{g}{h^3}\frac{4\pi m_\nu^3 v^2\,dv}{1+\exp\left(\frac{m_\nu v c}{\left(\frac{4}{11}\right)^{\frac{1}{3}}kT_0}\right)}\\
&= \frac{4\pi m_\nu^3 g}{h^3}\int_0^\infty \frac{v^2\,dv}{1+\exp\left(\frac{m_\nu v c}{\left(\frac{4}{11}\right)^{\frac{1}{3}}kT_0}\right)}\\
&= \frac{4\pi m_\nu^3 g}{h^3}\left(\frac{\left(\frac{4}{11}\right)^{\frac{1}{3}} kT_0}{m_\nu c}\right)^32!\zeta(3)\left(\frac{3}{4}\right)
\end{align*}

Then
\[\langle v\rangle = \frac{7\zeta(4)}{2\zeta(3)}\left(\frac{\left(\frac{4}{11}\right)^{\frac{1}{3}}k T_0}{m_\nu c}\right)\]

where $T_0 = 2.725\,K$

\section*{Problem 2}
\subsection*{a)}
In class, we showed that
\[Y_{He} = \frac{2\left(\frac{n_n}{n_p}\right)}{1+\frac{n_n}{n_p}}\]
Therefore, we show how the extra neutrino species affects the $\frac{n_n}{n_p}$ ratio at neutrino freeze-out. 

We consider the extra neutrino species as adding to the radiation density in the universe. As weak decoupling occurred during the radiation-dominated era, then this extra neutrino species would influence the expansion of the universe. Therefore, we examine how this affects the solution to $\Gamma_w = H$, the time of decoupling and the $\frac{n_n}{n_p}$ ratio.

In a radiation-dominated universe, $a(t)\propto t^{\frac{1}{2}}\implies$
\[\Gamma_w = n_0c\sigma_w\propto a^{-5}\propto t^{-\frac{5}{2}}\]
and from the Friedmann equation, assuming a flat universe,
\[H = \frac{H_0}{a^2}\sqrt{\Omega_\gamma} \propto \frac{H_0\sqrt{\Omega_\gamma}}{t}\]

Then let $t_d=$ time of decoupling.
\[C t^{-\frac{3}{2}} = H_0\sqrt{\Omega_\gamma}\]
where $C$ is some positive constant. Therefore, increasing $\Omega_\gamma$ with an extra neutrino species would cause $t_d$ to decrease. This would cause the temperature at decoupling to increase, and since
\[\frac{n_n}{n_p} = \left(\frac{n_n}{n_p}\right)^{\frac{3}{2}}e^{\frac{m_p-m_n}{kT}}\propto e^{-\frac{1}{T}}\]
Then $\frac{n_n}{n_p}$ increases as well.

As $Y_{He}$ is a monotonic function in $\frac{n_n}{n_p}$ for positive values, then this would cause $Y_{He}$ to increase.

\subsection*{b)}
As stated in part a), $\frac{n_n}{n_p}\propto e^{-\frac{1}{kT}}$ so decreasing $T$ will decrease the $\frac{n_n}{n_p}$ ratio and $Y_{He}$. We can also verify this by comparing against the decoupling value stated in class, $T_d = 9\times 10^9 K = .776 MeV/k$. We see that the new ratio should be $E^{-\frac{1.3\times 10^6}{.25\times 10^6}+\frac{1.3\times 10^6}{.776\times 10^6}}=	.029$ times the original $\frac{n_n}{n_p}$ ratio and therefore around a similar scale of difference between the new $Y_{He}$ and the original from the standard model.

\subsection*{c)}
Let $Q=m_p-m_n$. From class we have that $\frac{n_n}{n_p} \propto E^{-\frac{q}{kT}}$ up to the time of neutrino decoupling. Therefore, increasing $q$ will decrease $\frac{n_n}{n_p}$ and also decrease $Y_{He}$ since $Y_{He}$ is monotonic in $\frac{n_n}{n_p}$ at freeze-out.

\section*{Problem 3}
For a radiation dominated universe, $H = H_0 a^{-2}$. Then
\[t=H_0\int_0^a a'\,da'\implies a(t) = \sqrt{2H_0 t}\]

As demonstrated in class, for the radiation dominated universe,
\[\rho_\gamma\propto a^{-4}\propto T^4\implies T\propto a^{-4}\]

Therefore, we can can scale back from the observed $2.725\,K$ to $t=300\,s$. That is,
\[T_{300} = T_{now}a_{300}^{-1}=\frac{T_{now}}{\sqrt{2H_0 t}} = \frac{2.725\,K(\sqrt{3.086\times 10^{19}\frac{km}{Mpc}})}{\sqrt{(2)(70\frac{km}{Mpc\cdot s})(300\,s)}} = 7.39\times 10^7\,K=6.36keV\]

Since the protons, neutrons, and deuterium are in equilibrium, we can use the Saha equation to find an expression for
\[\frac{n_D}{n_pn_n} = \frac{g_D}{g_pg_n}\left(\frac{m_D}{m_pm_n}\right)^{\frac{3}{2}}\left(\frac{T}{2\pi}\right)^{-\frac{3}{2}}e^{Q_D}{T}\]
Let $m_D\approx 2m_p$ and $m_n\approx m_p$. Moreover, let $n_b$ be the number density of baryons so $n_p = .86n_b$, $n_n=.14n_b$. Then
\[\frac{n_D}{n_n} = .84n_b\frac{g_D}{g_pg_n}\left(\frac{4\pi}{m_p T}\right)^{\frac{3}{2}}e^{\frac{Q_D}{T}}\]

From class we also have
\[n_b = \eta n_\gamma \approx \eta (.2)T^3\]
\[\eta \approx 5\times 10^{-10}\frac{\Omega_b h^2}{.02} = 5\times 10^{-10}\]
Therefore, since we have 1 neutron per deuterium,
\[\frac{n_D}{n_n} = (.86)(.2)\eta \frac{g_D}{g_pg_n}\left(\frac{4\pi T}{m_p}\right)^{\frac{3}{2}}e^{\frac{Q_D}{T}} = (.84)(.2)(5\times 10^{-10})\frac{3}{(2)(2)}\left(\frac{4\pi (6.36keV)}{938\,MeV}\right)^{\frac{3}{2}}e^{\frac{-2.2MeV}{6.36keV}}=3.00\times 10^{-167}\]
\end{document}
We see that very few of the neutrons are in deuterium. 